\documentclass[aspectratio=169]{beamer}

\usepackage[utf8]{inputenc}
\usepackage{textcomp}
\usepackage[polish]{babel}
\usepackage{amsthm}
\usepackage{graphicx}
\usepackage[T1]{fontenc}
\usepackage{scrextend}
\usepackage{hyperref}
\usepackage{xcolor}
\usepackage{geometry}
\usepackage{listings}
\usepackage{epigraph}

\renewcommand{\epigraphsize}{\scriptsize}

\usetheme{-bjeldbak/beamerthemebjeldbak}

\definecolor{xbarcolor}{HTML}{0f2f0f}
\setbeamercolor{lower separation line head}{bg=xbarcolor} 
\setbeamercolor{lower separation line foot}{bg=xbarcolor} 

\title{Podstawy programownia (w języku C++)}
\subtitle{Wstęp do C++}
\author{Marek Marecki}
\institute{Polsko-Japońska Akademia Technik Komputerowych}

\lstset{basicstyle=\ttfamily\color{black},
columns=fixed,
escapeinside={\%*}{*)},
inputencoding=utf8,
extendedchars=true,
moredelim=**[is][\color{olive}]{$}{$},
moredelim=**[is][\color{red}]{@}{@}}

\begin{document}

{%
    \setbeamertemplate{headline}{}
    \frame{\titlepage}
}

\section{C++}

\begin{frame}
    \frametitle{Język C++}

    \begin{epigraphs}
        \qitem{Piękno, harmonia, wdzięk, dobry rytm - wszystkie zależą od
        prostoty.}{Platon, \emph{Republika}}

        \qitem{If you think it's simple, then you have misunderstood the
        problem.}{Bjarne Stroustrup, autor języka C++}
    \end{epigraphs}

    C++ jest rozbudowanym językiem oferującym wiele możliwości. Ceną za to jest
    jego ogromne skomplikowanie, brak ``elegancji'', i pułapki czychające na
    nieuważnego programistę.
\end{frame}

\section{Control flow}

\begin{frame}
    \frametitle{sequence}
    \framesubtitle{Control flow}

    Kontrola w języku C++ zaczyna się od pierwszej instrukcji w funkcji
    {\tt main()}. Kolejne instrukcje wykonywane są w kolejności zdefiniowanej w
    kodzie źródłowym, a większość z nich zakończona jest operatorem \textbf{{\tt ;}}

    \vspace{1em}

    Grupy instrukcji są ograniczane nawiasami klamrowymi \textbf{{\tt \{ \}}} i
    traktowane jako pojedyncza (ale nie atomowa!) instrukcja.

    \vspace{1em}

    Wewnątrz instrukcji, kolejnością wykonania steruje również operator
    \textbf{{\tt ,}}
\end{frame}

\begin{frame}[fragile]
    \frametitle{sequence}
    \framesubtitle{Control flow}

    \begin{lstlisting}
    auto x = 42@;@    // pierwsza instrukcja
    auto y = x@;@     // druga instrukcja
    print(y)@;@       // trzecia instrukcja
    \end{lstlisting}
\end{frame}

\begin{frame}
    \frametitle{selection}
    \framesubtitle{Control flow}

    \begin{labeling}{{\tt if-else}}
        \item[{\tt if-else}] wybiera następną instrukcję do wykonania na
            podstawie wartości logicznej dowolnego wyrażenia
        \item[{\tt switch}] wybiera nastepną instrukcję do wykonania na
            podstawie wartości typu wyliczeniowego\footnote{nie jest to do końca
            prawda -- można switchować na dowolnej wartości liczbowej, ale nie
            jest to dobry pomysł}(\emph{enum})
    \end{labeling}
\end{frame}

\begin{frame}[fragile]
    \frametitle{selection ({\tt if})}
    \framesubtitle{Control flow}

    \begin{lstlisting}
    if (x < y) {
        std::cout << "x less than y\n";
    } else if (x > y) {
        std::cout << "x greater than y\n";
    } else if (x == y) {
        std::cout << "x equals y\n";
    } else {
        std::cout << "something else\n";
    }
    \end{lstlisting}
\end{frame}

\begin{frame}
    \frametitle{selection ({\tt if} -- operatory)}
    \framesubtitle{Control flow}
    \label{selection_comparison_ops}

    Operatory porównania (poprzedni slajd) pozwalają na zbadanie relacji między
    dwiema wartościami:

    \begin{labeling}{$a$ {\tt ==} $b$}
        \item[$a$ {\tt <} $b$] czy $a$ jest mniejsze od $b$
        \item[$a$ {\tt >} $b$] czy $a$ jest większe od $b$
        \item[$a$ {\tt <=} $b$] czy $a$ jest mniejsze od lub równe $b$
        \item[$a$ {\tt >=} $b$] czy $a$ jest większe od lub równe $b$
        \item[$a$ {\tt !=} $b$] czy $a$ jest różne od $b$
        \item[$a$ {\tt ==} $b$] czy $a$ jest równe $b$
    \end{labeling}
\end{frame}

\begin{frame}[fragile]
    \frametitle{selection ({\tt switch})}
    \framesubtitle{Control flow}

    \begin{lstlisting}
    switch (x) {
        case Maybe::Something:
            std::cout << "it's something\n";
            break;
        case Maybe::Nothing:
            std::cout << "it's nothing\n";
            break;
        default:
            std::cout << "it's weird\n";
            break;
    }
    \end{lstlisting}
\end{frame}

\begin{frame}
    \frametitle{iteration}
    \framesubtitle{Control flow}

    \begin{labeling}{{\tt do-while}}
        \item[{\tt while}] pętla wykonująca się dopóki wyrażenie kontrolne jest
            prawdziwe, ze sprawdzeniem przed wykonaniem instrukcji
        \item[{\tt do-while}] pętla wykonująca się dopóki wyrażenie kontrolne
            jest prawdziwe, ze sprawdzeniem po wykonaniu instrukcji
        \item[{\tt for}] pętla po zakresie
    \end{labeling}

    \vspace{1em}

    Pętla {\tt while} jest najbardziej ogólną pętlą, ale wszystkie rodzaje są
    równoważne (każdą z pętli da się zaimplementować w ramach każdej innej).
\end{frame}

\begin{frame}[fragile]
    \frametitle{iteration ({\tt while})}
    \framesubtitle{Control flow}

    \begin{lstlisting}
    while (its_sunny_outside()) {
        std::cout << "weather is nice\n";
    }
    \end{lstlisting}
\end{frame}

\begin{frame}[fragile]
    \frametitle{iteration ({\tt do-while})}
    \framesubtitle{Control flow}

    \begin{lstlisting}
    auto x = 0;
    do {
        x = roll_dice();
        std::cout << "you rolled " << x << "\n";
    } while (x != 6);
    \end{lstlisting}
\end{frame}

\begin{frame}[fragile]
    \frametitle{iteration ({\tt for})}
    \framesubtitle{Control flow}

    \begin{lstlisting}
    for (auto i = 10; i >= 0; --i) {
        std::cout << i << '\n';
    }
    std::cout << "Happy New Year!\n";
    \end{lstlisting}
\end{frame}

\begin{frame}
    \frametitle{procedural abstraction}
    \framesubtitle{Control flow}

    Funkcje spełniają rolę procedur w C++. Każda funkcja składa się z:

    \begin{enumerate}
        \item nazwy -- identyfikatora jakim można ją \emph{wywołać}
        \item listy parametrów formalnych -- specyfikacji jakich
            \emph{argumentów} (\emph{parametrów faktycznych}) wymaga od
            wywołującego
        \item typu zwracanego -- specyfikacji typu jakiego wartości funkcja
            produkuje
        \item ciała -- ograniczonego nawiasami klamrowymi zbioru instrukcji
            określającego operacje jakich dana funkcja jest abstrakcją
    \end{enumerate}
\end{frame}

\begin{frame}[fragile]
    \frametitle{procedural abstraction}
    \framesubtitle{Control flow}

    \begin{lstlisting}
    auto add_one(int const x) -> int
    {
        return (x + 1);
    }
    \end{lstlisting}
\end{frame}

\begin{frame}
    \frametitle{recursion}
    \framesubtitle{Control flow}

    Rekurencja jest realizowana za pomocą funkcji.
\end{frame}

\begin{frame}[fragile]
    \frametitle{recursion}
    \framesubtitle{Control flow}

    \begin{lstlisting}
    /* Raises b to the power of n. */
    auto exponentiate(int const b, int const n) -> int
    {
        if (n <= 0) {
            return 1;
        }
        return (b * exponentiate(b, (n - 1)));
    }
    \end{lstlisting}
\end{frame}

\begin{frame}
    \frametitle{concurrency and parallelism}
    \framesubtitle{Control flow}

    Współbieżność w C++ jest realizowana za pomocą \emph{wątków}. Tym samym
    mechanizmem jest realizowana \emph{równoległość} przetwarzania
    (\emph{parallelism}).

    \vspace{1em}

    Współbieżność można też zaimplementować na własną rękę, ale wymaga to
    znacznie większego nakładu pracy.
\end{frame}

\begin{frame}[fragile]
    \frametitle{concurrency and parallelism ({\tt std::thread})}
    \framesubtitle{Control flow}

    {\footnotesize
    \begin{lstlisting}
    auto display_greeting(std::string const name) -> void
    {
        std::cout << ("Hello, " + name + "!\n");
    }

    auto t1 = std::thread{display_greeting, "Joe"};    // Armstrong
    auto t2 = std::thread{display_greeting, "Bjarne"}; // Stroustrup

    /*
     * Threads must be joined into the parent thread, or
     * the program will crash.
     */
    t1.join();  // joining thread is blocked until joined
                // thread terminates
    t2.join();
    \end{lstlisting}}
\end{frame}

\begin{frame}
    \frametitle{nondeterminism}
    \framesubtitle{Control flow}

    Niedeterminizm jest nieodłączną cechą równoległości -- nie mamy gwarancji w
    jakiej kolejności będą względem siebie wykonywac sie operacje w
    \emph{różnych} wątkach.

    \vspace{1em}

    Niedeterminizm wewnątrz wątku możemy uzyskać generując liczby losowe. W tym
    celu można użyć {\tt std::ranom\_device} lub odczytać $n$ bajtów z pliku
    {\tt /dev/urandom}.
\end{frame}

\begin{frame}[fragile]
    \frametitle{nondeterminism ({\tt std::random\_device})}
    \framesubtitle{Control flow}

    {\footnotesize
    \begin{lstlisting}
    std::random_device rd;
    std::uniform_int_distribution<int> d20 (1, 20);

    constexpr auto CRITICAL_SUCCESS = 20;
    constexpr auto CRITICAL_FAILURE = 1;

    auto const x = d20(rd);

    if (x == CRITICAL_SUCCESS) {
        std::cout << "you kill the monster in a single blow!\n";
    } else if (x == CRITICAL_FAILURE) {
        std::cout << "you wound yourself with your own sword!\n";
    } else {
        std::cout << "roll for damage.\n";
    }
    \end{lstlisting}}
\end{frame}

\begin{frame}
    \frametitle{exceptions}
    \framesubtitle{Control flow}

    Mechanizmem dedykowanym sygnalizacji i obsługi błędów w C++ są
    \emph{wyjątki}. Wyjątek może być rzucony (zasygnalizowany) słowem kluczowym
    {\tt throw}; obsługa wyjątków odbywa się w bloku {\tt try-catch}.

    \vspace{1em}

    C++ pozwala na użycie dowolnego typu jako wyjątku.
\end{frame}

\begin{frame}[fragile]
    \frametitle{exceptions}
    \framesubtitle{Control flow}

    {\footnotesize
    \begin{lstlisting}
    auto search_your_feelings(std::string father) -> void
    {
        if (father == "Darth Vader") {
            throw std::string{"NO!!! NO!!!"};
        }
    }

    /* ... */

    try {
        luke.search_your_feelings(lord_vader);
    } catch (std::string const& error) {
        std::cerr << ("operation failed: " + error + '\n');
    }
    \end{lstlisting}}
\end{frame}

\section{Data types}

\begin{frame}
    \frametitle{fundamentalne typy danych}
    \framesubtitle{Data structures}

    \begin{labeling}{{\tt double}}
        \item[{\tt void}] typ reprezentujący ``nic''
        \item[{\tt bool}] typ reprezentujący wartości logiczne
        \item[{\tt char}] typ reprezentujący znaki
        \item[{\tt int}] typ reprezentujący liczby całkowite
        \item[{\tt double}] typ reprezentujący liczby zmiennoprzecinkowe
    \end{labeling}

    \vspace{1em}

    \url{https://en.cppreference.com/w/cpp/language/types}
\end{frame}

\begin{frame}[fragile]
    \frametitle{{\tt void}}
    \framesubtitle{fundamentalne typy danych}

    Typ {\tt void} jest często używany jeśli funkcja nie produkuje żadnej
    wartości\footnote{takie funkcje nazywane są czasem ``procedurami'', ang.
    \emph{procedure}}:

    \begin{lstlisting}
    auto produce_nothing() -> void
    {}
    \end{lstlisting}
\end{frame}

\begin{frame}[fragile]
    \frametitle{{\tt bool}}
    \framesubtitle{fundamentalne typy danych}

    Typ {\tt bool} jest zwracany przez operatory porównania (slajd
    \pageref{selection_comparison_ops}). Jest również typem ``argumentu''
    instrukcji {\tt if}.
    Przyjmuje jedynie dwie wartości -- reprezentowane słowami kluczowymi {\tt
    true} i {\tt false}:

    \begin{lstlisting}
    auto you_know_it_to_be = true;
    auto lies_and_deception = false;
    \end{lstlisting}
\end{frame}

\begin{frame}[fragile]
    \frametitle{{\tt char}}
    \framesubtitle{fundamentalne typy danych}

    Typ {\tt char} jest używany do reprezentowania znaków. Wywodzi się z czasów
    kiedy królowało kodowanie ASCII, a znaki (oraz kody kontrolne) zajmowały
    sztywne 7 bitów.

    \begin{lstlisting}
    auto a = 'A';
    auto new_line = '\n';
    \end{lstlisting}

    Domyślnie typ {\tt char} reprezentuje wartości ``ze znakiem'' czyli de~facto
    {\tt signed~char}; możliwe jest wymuszenie ``bezznakowości'' pisząc {\tt
    unsigned~char}.
\end{frame}

\begin{frame}[fragile]
    \frametitle{{\tt int}}
    \framesubtitle{fundamentalne typy danych}

    Typ {\tt int} jest używany do reprezentowania liczb całkowitych, domyślnie
    ze znakiem:

    \begin{lstlisting}
    auto answer_to_everything  = 42;
    auto negative_of_the_beast = -666;
    \end{lstlisting}

    Typ {\tt int} można ``doprecyzować'' specyfikatowami znaku i rozmiaru:

    \begin{labeling}{{\tt unsigned}}
        \item[{\tt signed}] liczba całkowita ze znakiem (domyślnie)
        \item[{\tt unsigned}] liczba całkowita bez znaku
        \item[{\tt short}] ``krótka'' liczba, czyli zazwyczaj 16 bitów
        \item[{\tt long}] ``długa'' liczba, czyli zazwyczaj 64 bity
    \end{labeling}

    Specyfikatory można łączyć, np. w tym {\tt unsigned long int}.
\end{frame}

\begin{frame}[fragile]
    \frametitle{{\tt double}}
    \framesubtitle{fundamentalne typy danych}

    Typ {\tt double} jest używany do reprezentowania liczb zmiennoprzecinkowych.
    Jego mniej precyzyjną alternatywą jest typ {\tt float}.

    \begin{lstlisting}
    auto pi     = 3.14f;   // float
    auto piiiii = 3.14159; // double
    \end{lstlisting}

    Domyślnym typem literałów zmiennoprzecinkowych jest {\tt double}.
\end{frame}

\begin{frame}
    \frametitle{złożone typy danych}
    \framesubtitle{Data structures}

    Złożone (ang. \emph{compound}) typy danych składają się z typu
    podstawowego i ``dodatku'':

    \begin{labeling}{{\tt T const}}
    \item[{\tt T[\emph{n}]}] tablica {\tt \emph{n}} wartości typu {\tt T}
    \item[{\tt T*}] wskaźnik do obszaru pamięci zawierającego wartość
        typu {\tt T}
    \item[{\tt T\&}] referencja do wartości typu {\tt T}
    \item[{\tt T const}] stała wartość typu {\tt T}
    \end{labeling}

    \vspace{1em}

    \url{https://en.cppreference.com/w/cpp/language/type}
\end{frame}

\begin{frame}[fragile]
    \frametitle{tablica}
    \framesubtitle{złożone typy danych}

    Tablice, odziedziczone z języka C, służą do przechowywania kilku wartości
    tego samego typu:
    {\footnotesize \begin{lstlisting}
    int decimal_digits[10] = { 0, 1, 2, 3, 4, 5, 6, 7, 8, 9 };
    \end{lstlisting}}

    Tablica może mieć również rozmiar nieokreślony -- widać to np. w funkcji
    {\tt main()}:

    {\footnotesize \begin{lstlisting}
    auto main(int argc, char* argv[]) -> int
    {
        return 0;
    }
    \end{lstlisting}}
\end{frame}

\begin{frame}[fragile]
    \frametitle{wskaźnik i referencja}
    \framesubtitle{złożone typy danych}

    Wskaźniki, odziedziczone z języka C, służą do przechowywania adresów
    obszarów pamięci zawierających wartości typu {\tt T}:
    {\footnotesize \begin{lstlisting}
    auto a_value   = 42; $// or, explicitly: int a_value ...$
    auto a_pointer = &a; $// or, explicitly: int* a_pointer ...$
    \end{lstlisting}}

    Referencje służą do przechowywania odniesień do wartości typu {\tt T}:
    {\footnotesize \begin{lstlisting}
    auto  a_value = 42; $// or, explicitly: int a_value ...$
    auto& a_ref   = a;  $// or, explicitly: int& a_pointer ...$
    \end{lstlisting}}

    Wskaźniki mogą być ``zerowe'' tzw. \emph{null pointer} i nie wskazywać na
    poprawny adres. Referencje \emph{zawsze} wskazują na istniejącą
    wartość\footnote{przynajmniej w teorii...}.
\end{frame}

\subsection{Data structures}

\begin{frame}
    \frametitle{biblioteka standardowa}
    \framesubtitle{Data structures}

    Biblioteka standardowa (ang. \emph{standard library}) języka C++ zawiera
    wiele struktur danych takich jak:

    \begin{labeling}{{\tt std::vector}}
    \item[{\tt std::array}] ``silna'' tablica
    \item[{\tt std::vector}] tablica o zmiennej długości
    \item[{\tt std::queue}] kolejka FIFO
    \item[{\tt std::map}] struktura mapująca klucze typu $T$ na wartości typu $V$
    \item[{\tt std::string}] napis
    \item[{\tt std::pair}] para wartości typów $F$ i $S$
    \end{labeling}

    \vspace{1em}

    Warto używać struktur z biblioteki standardowej żeby oszczędzić sobie pracy.
\end{frame}

\begin{frame}
    \frametitle{własne typy danych}
    \framesubtitle{Data structures}

    Programista C++ może definiować również własne typy danych: struktury i
    klasy, oraz wyliczenia.

    \vspace{1em}

    Klasy ({\tt class}) różnią się od struktur ({\tt struct}) tylko i wyłącznie
    tym, że ich pola są domyślnie publiczne.

    \vspace{1em}

    Wyliczenia słabe ({\tt enum}) są typu {\tt int}, ich wartości są globalne, i
    mają automatycznie zdefiniowane operacje arytmetyczne (np. sumę bitową). Są
    przydatne przy definiowaniu flag, które można łączyć.

    Wyliczenia silne ({\tt enum class}) różnią sie od słabych tym, że są
    ''swojego własnego'' typu, ich wartości nie są globalne, oraz nie mają
    automatycznie zdefiniowanych operacji arytmetycznych. Są przydatne przy
    definiowaniu rozdzielnych stanów.
\end{frame}

\begin{frame}[fragile]
    \frametitle{struktury ({\tt struct})}
    \framesubtitle{Data structures}

    {\scriptsize
    \begin{lstlisting}
    struct being_with_legs {
        std::string const name;
        size_t const legs;

        being_with_legs(std::string, size_t);
    };

    being_with_legs::being_with_legs(std::string n, size_t l)
        : name{std::move(n)}
        , legs{l}
    {}

    /* ... */

    auto const snake = being_with_legs{ "snake", 0 };
    auto const human = being_with_legs{ "human", 2 };
    auto const spider = being_with_legs{ "spider", 8 };
    \end{lstlisting}}
\end{frame}

\begin{frame}[fragile]
    \frametitle{wyliczenia ``silne'' ({\tt enum class})}
    \framesubtitle{Data structures}

    {\scriptsize
    \begin{lstlisting}
    enum class meal_kind {
        BREAKFAST,
        DINNER,
        SUPPER,
    };

    auto is_most_important_meal_of_the_day(meal_kind const meal) -> bool
    {
        return (meal == meal_kind::BREAKFAST);
    }
    \end{lstlisting}}
\end{frame}

\begin{frame}[fragile]
    \frametitle{wyliczenia ({\tt enum})}
    \framesubtitle{Data structures}

    {\scriptsize
    \begin{lstlisting}
    enum some_flags_type {
        SOME_FLAG_READ,
        SOME_FLAG_WRITE,
        SOME_FLAG_NONBLOCK,
        SOME_FLAG_BUFFERED,
        SOME_FLAG_UNBUFFERED,
    };
    constexpr auto SOME_FLAG_DEFAULT = SOME_FLAG_READ
                                     | SOME_FLAG_WRITE
                                     | SOME_FLAG_BUFFERED;

    // We want read-only, non-blocking, unbuffered descriptor.
    auto const mode = SOME_FLAG_READ
                    | SOME_FLAG_NONBLOCK
                    | SOME_FLAG_UNBUFFERED;
    \end{lstlisting}}
\end{frame}

\section{I/O}

\begin{frame}
    \frametitle{I/O}

    C++ korzysta z mechanizmów I/O dostarczanych przez API systemu operacyjnego
    (np. Linux), ale część z nich opakowuje w swoje własne abstrakcje
    zapewniając programom przenośność.
\end{frame}

\begin{frame}
    \frametitle{standard steams}
    \framesubtitle{I/O}

    W momencie uruchomienia dla większości programów tworzone są 3 standardowe
    strumienie: wejścia, wyjścia, i błędów.

    \begin{labeling}{{\tt std::cerr}}
    \item[{\tt std::cin}] standardowy strumień wejścia, służy do odczytu danych
        podawanych przez użytkownika w konsoli tekstowej (\emph{file descriptor
        0})
    \item[{\tt std::cout}] standardowy strumień wyjścia, służy do prezentacji
        wyników działania programu w konsoli tekstowej (\emph{file descriptor
        1})
    \item[{\tt std::cerr}] standardowy strumień błędów, służy do prezentacji
        błędów działania i awarii programu w konsoli tekstowej (\emph{file
        descriptor 2})
    \end{labeling}
\end{frame}

\begin{frame}[fragile]
    \frametitle{standard steams}
    \framesubtitle{I/O}

    {\scriptsize
    \begin{lstlisting}
    {
        std::string line;

        // read a line of text from standard input
        std::getline(std::cin, line);
    }

    // display a message to inform user of what is happening...
    std::cerr << "connecting to server...\n";

    // ...or notify them about errors
    std::cerr << "connection failed\n";

    // display results of program's work
    std::cout << downloaded_data << '\n';
    \end{lstlisting}}
\end{frame}

\begin{frame}
    \frametitle{files}
    \framesubtitle{I/O}

    C++ definiuje typy {\tt std::ifstream} (\emph{input file stream}) i
    {\tt std::ofstream} (\emph{output file stream}) w bibliotece standardowej.

    \vspace{1em}

    Jeśli ich interfejs nie jest wystarczający zawsze można użyć interfejsu
    platformy czyli np. wywołań systemowych definiowanych przez standard POSIX --
    {\tt open(3)}, {\tt write(3)}, {\tt read(3)}, i {\tt close(3)}.
\end{frame}

\begin{frame}[fragile]
    \frametitle{files ({\tt std::ifstream} and {\tt std::ofstream})}
    \framesubtitle{I/O}

    {\scriptsize
    \begin{lstlisting}
    auto path = std::string{"./data.txt"};

    {
        // write line to a file
        auto out = std::ofstream{ path };
        if (out.good()) {
            out << "Hello, World!\n";
        }
    }

    {
        // read line from a file
        auto in = std::ifstream{ path };
        if (in.good()) {
            auto line = std::string{};
            std::getline(in, line);
            std::cout << line << "\n";
        }
    }
    \end{lstlisting}}
\end{frame}

\end{document}

% \begin{frame}
%     \frametitle{sieć}
%     \framesubtitle{I/O}

%     Komunikacja po sieci odbywa się z wykorzystaniem mechanizmów I/O
%     dostarczanych przez standard POSIX --
%     {\tt socket(2)},
%     {\tt bind(2)},
%     {\tt listen(2)},
%     {\tt accept(2)},
%     {\tt connect(2)},
%     {\tt inet\_pton(3)},
%     {\tt connect(3)},
%     {\tt shutdown(3)},
%     {\tt close(3)}.
% \end{frame}

% \begin{frame}[fragile]
%     \frametitle{sieć (nagłówki)}
%     \framesubtitle{I/O}

%     Pliki nagłówkowe, które należy dodać na początku pliku z kodem źródłowym
%     żeby móc bez przeszkód korzystać z funkcji pozwalających na komunikację po
%     sieci:

%     {\small
%     \begin{lstlisting}
%     #include <arpa/inet.h>
%     #include <endian.h>
%     #include <netinet/ip.h>
%     #include <string.h>
%     #include <sys/types.h>
%     #include <sys/socket.h>
%     #include <sys/un.h>
%     #include <unistd.h>
%     \end{lstlisting}}
% \end{frame}

% \begin{frame}[fragile]
%     \frametitle{sieć (klient)}
%     \framesubtitle{I/O}

%     {\scriptsize
%     \begin{lstlisting}
%     auto sock = socket(AF_INET, SOCK_STREAM, 0);

%     auto const ip = std::string{"127.0.0.1"};
%     auto const port = uint16_t{42420};

%     sockaddr_in server;
%     memset(&server, 0, sizeof(server));
%     server.sin_family = AF_INET;
%     server.sin_port = htobe16(port);
%     inet_pton(AF_INET, ip.c_str(), reinterpret_cast<void*>(&server.sin_addr));

%     connect(sock, reinterpret_cast<sockaddr*>(&server), sizeof(server));

%     auto const data = std::string{"Hello, World!"};
%     write(sock, data.c_str(), data.size());

%     shutdown(sock, SHUT_RDWR);
%     close(sock);
%     \end{lstlisting}}
% \end{frame}

% \begin{frame}[fragile]
%     \frametitle{sieć (serwer, część $\frac{1}{2}$)}
%     \framesubtitle{I/O}

%     {\tiny
%     \begin{lstlisting}
%     auto sock = socket(AF_INET, SOCK_STREAM, 0);

%     auto const ip = std::string{"127.0.0.1"};
%     auto const port = uint16_t{42420};

%     sockaddr_in server;
%     memset(&server, 0, sizeof(server));

%     server.sin_family = AF_INET;
%     server.sin_port = htobe16(port);

%     inet_pton(AF_INET, ip.c_str(), reinterpret_cast<void*>(&server.sin_addr));

%     bind(sock, reinterpret_cast<sockaddr*>(&server), sizeof(server));
%     listen(sock, 0);  // default limit of waiting connections
%     \end{lstlisting}}

%     W ten sposób przygotuje się socket do odbierania połączeń.
% \end{frame}

% \begin{frame}[fragile]
%     \frametitle{sieć (serwer, część $\frac{2}{2}$)}
%     \framesubtitle{I/O}

%     {\tiny
%     \begin{lstlisting}
%     sockaddr_in client_addr;
%     memset(&client_addr, 0, sizeof(client_addr));
%     auto client_len = socklen_t{sizeof(client_addr)};
%     auto client = accept(sock,
%         reinterpret_cast<sockaddr*>(&client_addr), &client_len);
%     {
%         // report client's address
%         std::array<char, INET_ADDRSTRLEN + 1> buffer {};
%         inet_ntop(AF_INET, &client_addr.sin_addr, buffer.data(), buffer.size());
%         std::cout << buffer.data() << ':' << be16toh(client_addr.sin_port) << "\n";
%     }
%     {
%         // read data from client
%         std::array<char, 512> buffer {};
%         auto const n = read(client, buffer.data(), buffer.size());
%         std::cout << std::string{buffer.data(), buffer.data() + n} << "\n";
%     }

%     shutdown(sock, SHUT_RDWR);
%     close(sock);
%     \end{lstlisting}}

%     W ten sposób odbiera się połączenia i rejestruje adres połączonego klienta.
% \end{frame}
