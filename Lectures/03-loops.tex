\documentclass[aspectratio=169]{beamer}

\usepackage[utf8]{inputenc}
\usepackage{textcomp}
\usepackage[polish]{babel}
\usepackage{amsthm}
\usepackage{graphicx}
\usepackage[T1]{fontenc}
\usepackage{scrextend}
\usepackage{hyperref}
\usepackage{xcolor}
\usepackage{geometry}
\usepackage{listings}
\usepackage{epigraph}

\renewcommand{\epigraphsize}{\scriptsize}

\usetheme{-bjeldbak/beamerthemebjeldbak}

\definecolor{xbarcolor}{HTML}{000000}
\setbeamercolor{lower separation line head}{bg=xbarcolor}
\setbeamercolor{lower separation line foot}{bg=xbarcolor}

\title{Podstawy programownia (w języku C++)}
\subtitle{Pętle}
\author{Marek Marecki}
\institute{Polsko-Japońska Akademia Technik Komputerowych}

\lstset{basicstyle=\ttfamily\color{black},
columns=fixed,
escapeinside={[*}{*]},
inputencoding=utf8,
extendedchars=true,
moredelim=**[is][\color{red}]{@}{@},
moredelim=**[is][\color{gray}]{`}{`},
moredelim=**[is][\color{olive}]{$}{$}}

\begin{document}

{%
    \setbeamertemplate{headline}{}
    \frame{\titlepage}
}

\section{Tablica: C-style vs C++}

\begin{frame}
    \frametitle{Po co?}

    Tablica oraz jej pochodne takie jak {\tt std::array} czy {\tt std::vector}
    są sposobem na przechowywanie więcej niż 1 wartości (tego samego typu) w
    jednej zmiennej.

    \vspace{1em}

    Upraszcza to programowanie pozwalając na wygodną iterację po elementach
    takich \emph{sekwencji} zamiast np. analizowaniu 100 zmiennych z osobna, czy
    umożliwiając przechowywanie zmiennych ilości elementów -- niekoniecznie
    znanych na etapie pisania programu.
\end{frame}

\begin{frame}
    \frametitle{tablica (C-style array)}

    Tablica w stylu C jest obszarem pamięci o stałym rozmiarze pozwalającym na
    przechowanie \emph{n} elementów, gdzie \emph{n} musi być znane na etapie
    kompilacji\footnote{{\tt constexpr} w nomenklaturze C++}.
\end{frame}

\begin{frame}
    \frametitle{wady}
    \framesubtitle{tablica (C-style array)}

    Wadą tablicy w stylu C jest to, że bardzo łatwo jest utracić informację o
    jej rozmiarze ponieważ ona sama ma do niego bardzo ``luźne'' podejście i de
    facto nie zawiera takiej informacji.
    Tablica może być \emph{automatycznie rzutowana} na wskaźnik, a po takiej
    operacji niemal niemożliwe jest odzyskanie informacji o tym ile elementów
    zawiera.

    \vspace{1em}

    Brak informacji (lub niepoprawna informacja) o rozmiarze tablicy może
    prowadzić do tzw. \emph{buffer overflow} i, pozwalając na nadpisanie
    niepowiązanych z tablicą obszarów pamięci, prowadzić do usterek w
    programie.
\end{frame}

\begin{frame}[fragile]
    \frametitle{inicjalizacja tablic w stylu C}
    \framesubtitle{tablica (C-style array)}

    {\footnotesize
    \begin{lstlisting}
    `#include <algorithm>`

    int numbers[100];  $// uninitialised array of 100 integers$
                       $// its elements contain "random garbage"$

    $// each element holds a zero$
    std::fill_n(numbers, 100, 0);
    \end{lstlisting}}
\end{frame}

\begin{frame}[fragile]
    \frametitle{inicjalizacja tablic w stylu C (2)}
    \framesubtitle{tablica (C-style array)}

    {\footnotesize
    \begin{lstlisting}
    `#include <algorithm>`

    constexpr auto NUMBERS_SIZE = 100;
    int numbers[NUMBERS_SIZE];
    std::fill_n(numbers, NUMBERS_SIZE, 0);
    \end{lstlisting}}
\end{frame}

\begin{frame}[fragile]
    \frametitle{inicjalizacja tablic w stylu C (3)}
    \framesubtitle{tablica (C-style array)}

    Tablicę można również zainicjalizować podając jej elementy w nawiasach
    klamrowych. Rozmiar tablicy jest w takim przypadku określany automatycznie.

    {\footnotesize
    \begin{lstlisting}
    int numbers[] = { 1, 2, 4, 8, 16, 32, 64, 128 };

    constexpr auto NUMBERS_SIZE
        = (sizeof(numbers) / sizeof(numbers[0]));
    \end{lstlisting}}

    Używając operatora {\tt sizeof} można ``odzyskać'' rozmiar tablicy
    obliczając go.
\end{frame}

\begin{frame}[fragile]
    \frametitle{dostęp do wartości}

    Tablice są \emph{indeksowane} liczbami całkowitymi.\\
    Indeksem pierwszego elementu jest 0.

    {\footnotesize
    \begin{lstlisting}
    `int numbers[100];`

    numbers[0] = 42;  $// write an element$

    auto x = numbers[0];  $// read an element$
    \end{lstlisting}}

    W ten sam sposób można dostać się do elementów {\tt std::array} i {\tt
    std::vector}.
\end{frame}

\begin{frame}
    \frametitle{{\tt std::array}}

    {\tt std::array} jest strukturą danych pozwalającą na przechowanie \emph{n}
    elementów, gdzie \emph{n} musi być znane na etapie kompilacji i jest
    niezmienne w trakcie działania programu.
\end{frame}

\begin{frame}
    \frametitle{wady i zalety}
    \framesubtitle{{\tt std::array}}

    Zaletą {\tt std::array} względem tablicy w stylu C jest to, że jest ona
    osobnym typem danych, a nie prostym wskaźnikiem na obszar pamięci. Pozwala
    jej to m.in. na śledzenie swojego rozmiaru, oraz zapobiega przypadkowym
    automatycznym konwersjom.

    \vspace{1em}

    Zarówno wadą jak i zaletą jest jednak fakt, że typ tablicy zawiera liczbę
    jej elementów. Powoduje to, że C++ nie pozwoli przekazać do tego samego
    parametru zarówno {\tt std::array<int, 10>} oraz {\tt std::array<int, 100>}.
    Jeśli jedna funkcja ma obsługiwać tablice różnych rozmiarów to trzeba uciec
    się do szablonów, wskaźników, lub iteratorów.
\end{frame}

\begin{frame}[fragile]
    \frametitle{inicjalizacja}
    \framesubtitle{{\tt std::array}}

    {\footnotesize
    \begin{lstlisting}
    `#include <algorithm>`
    `#include <array>`

    $// uninitialised array of 100 integers$
    $// its elements contain "random garbage"$
    std::array<int, 100> numbers;

    $// initialise all elements to 0$
    std::fill(numbers.begin(), numbers.end(), 0);

    $// initialise all elements to 0$
    std::fill_n(numbers.begin(), numbers.size(), 0);
                                   
    \end{lstlisting}}
\end{frame}

\subsection{Wektor}

\begin{frame}
    \frametitle{{\tt std::vector}}

    {\tt std::vector} jest strukturą danych reprezentującą tablicę zmiennego
    rozmiaru.

    Podczas działania programu można do wartości typu {\tt std::vector} dodawać
    i usuwać elementy, a rozmiar wektora dostosuje się do aktualnej ich liczby.
\end{frame}

\begin{frame}
    \frametitle{wady i zalety}
    \framesubtitle{{\tt std::array}}

    Zaletą {\tt std::vector} względem tablic jest automatyczne dostosowywanie
    rozmiaru. Jeśli ilość elementów jaka może być wymagana nie jest znana na
    etapie kompilacji to {\tt std::vector} z łatwością taką sytuację obsłuży.

    \vspace{1em}

    Pewną wadą {\tt std::vector} jest nadmierne zużycie pamięci w niektórych
    przypadkach. Wektor podczas powiększania rozmiaru alokuje pamięć z
    naddatkiem, który nie musi być przez program użyty. W takiej sytuacji
    rozmiar wektora można ręcznie pomniejszyć do aktualnie wymaganego.
\end{frame}

\begin{frame}[fragile]
    \frametitle{inicjalizacja}
    \framesubtitle{{\tt std::array}}

    {\footnotesize
    \begin{lstlisting}
    `#include <vector>`

    $// empty vector of integers$
    auto ve = std::vector<int>{};

    $// vector of 100 zero-initialised integers$
    auto vd = std::vector<int>(100);

    $// vector of 4 integers$
    auto vd = std::vector<int>{ 2, 4, 8, 16 };
    \end{lstlisting}}
\end{frame}

\begin{frame}[fragile]
    \frametitle{dodawanie i usuwanie wartości}
    \framesubtitle{{\tt std::array}}

    {\footnotesize
    \begin{lstlisting}
    `#include <algorithm>`
    `#include <vector>`

    auto v = std::vector<int>{ -1, 0, 1 };

    $// push 42 to the vector as the last element$
    v.push_back(42);

    $// pop (remove) last element from the vector$
    v.pop_back();

    $// remove element with a specific value from the vector$
    v.erase(std::find(v.begin(), v.end(), -1));
    \end{lstlisting}}
\end{frame}

\begin{frame}[fragile]
    \frametitle{zmiana rozmiaru}
    \framesubtitle{{\tt std::array}}

    {\footnotesize
    \begin{lstlisting}
    `#include <vector>`

    auto v = std::vector<int>{};

    $// resize the vector to hold 42 elements$
    v.resize(42);

    $// remove all elements to the vector$
    v.clear();

    $// reduce capacity of the vector to match its size$
    v.shrink_to_fir();
    \end{lstlisting}}
\end{frame}

\section{{\tt while}, {\tt do-while}, {\tt for}}

\begin{frame}[fragile]
    \frametitle{Po co?}
    \framesubtitle{Pętla {\tt while}}

    Pętla {\tt while} sprawdza się kiedy instrukcja przez nią wykonywana powinna
    być powtarzana \emph{dopóki} pewien warunek jest spełniony.

    \begin{lstlisting}
    while (system_is_running()) {
        process_events();
    }
    \end{lstlisting}

    Istotne jest to, że warunek sprawdzany jest \emph{przed} wykonaniem
    instrukcji.
\end{frame}

\begin{frame}[fragile]
    \frametitle{Po co?}
    \framesubtitle{Pętla {\tt do-while}}

    Pętla {\tt do-while} sprawdza się kiedy instrukcja przez nią wykonywana powinna
    być powtarzana \emph{dopóki} pewien warunek jest spełniony, ale musi być
    wykonana \emph{co najmniej jeden raz}.

    \begin{lstlisting}
    do {
        process_events();
    } while (system_is_running());
    \end{lstlisting}

    Istotne jest to, że warunek sprawdzany jest \emph{po} wykonaniu
    instrukcji.
\end{frame}

\begin{frame}[fragile]
    \frametitle{Warunek}
    \framesubtitle{Pętle {\tt while} i {\tt do-while}}

    Warunek jest podawany w nawiasach po słowie kluczowym {\tt while}, i może
    być w zasadzie dowolny.

    \begin{lstlisting}
    while (@system_is_running()@) `{
        process_events();
    }`
    \end{lstlisting}

    albo

    \begin{lstlisting}
    `do {
        process_events();
    }` while (@system_is_running()@)`;`
    \end{lstlisting}
\end{frame}

\begin{frame}[fragile]
    \frametitle{Instrukcja}
    \framesubtitle{Pętle {\tt while} i {\tt do-while}}

    Instrukcja powtarzana przez pętlę jest podawana w nawiasach klamrowych:

    \begin{lstlisting}
    `while (system_is_running()) `{
        @process_events();@
    }
    \end{lstlisting}

    albo

    \begin{lstlisting}
    `do` {
        @process_events();@
    } `while (system_is_running());`
    \end{lstlisting}
\end{frame}

\begin{frame}[fragile]
    \frametitle{Ad infinitum}
    \framesubtitle{Pętle {\tt while} i {\tt do-while}}

    Do implementacji pętli nieskończonych często wykorzystuje się konstrukcję
    {\tt while-true}:

    \begin{lstlisting}
    while (@true@) `{
        process_events();
    }`
    \end{lstlisting}

    Pętle nieskończone są często spotykane w ''sercach'' długo działających
    programów (systemów operacyjnych, gier, itp.), których zakończenie jest
    wywoływane przez jakieś zewnętrzne zdarzenie (np. akcję użytkownika), a nie
    przez wewnętrzny stan programu (np. koniec danych do przetworzenia).
\end{frame}

\begin{frame}[fragile]
    \frametitle{Krok po kroku}
    \framesubtitle{Pętle {\tt while} i {\tt do-while}}

    \begin{lstlisting}
    while
    \end{lstlisting}
    \vspace{2.4em}

    vs

    \vspace{2.4em}
    \begin{lstlisting}
      while
    \end{lstlisting}
\end{frame}

\begin{frame}[fragile]
    \frametitle{Krok po kroku}
    \framesubtitle{Pętle {\tt while} i {\tt do-while}}

    \begin{lstlisting}
    `while` (condition_is_met())
    \end{lstlisting}
    \vspace{2.4em}

    vs

    \vspace{2.4em}
    \begin{lstlisting}
      `while` (condition_is_met());
    \end{lstlisting}
\end{frame}

\begin{frame}[fragile]
    \frametitle{Krok po kroku}
    \framesubtitle{Pętle {\tt while} i {\tt do-while}}

    \begin{lstlisting}
    `while (condition_is_met())` {
        take_action();  $// maybe never$
    }
    \end{lstlisting}

    vs

    \begin{lstlisting}
    do {
        take_action();  $// at least once$
    } `while (condition_is_met());`
    \end{lstlisting}
\end{frame}

\subsection{{\tt for}}

\begin{frame}[fragile]
    \frametitle{Po co?}
    \framesubtitle{Pętla {\tt for}}

    Pętla {\tt for} sprawdza się kiedy instrukcja przez nią wykonywana powinna
    być powtarzana pewną \emph{ilość razy} określoną przez licznik pętli.

    \begin{lstlisting}
    std::cout << argv[0];
    for (auto i = 1; i < argc; ++i) {
        std::cout << " " << argv[i];
    }
    std::cout << "\n";
    \end{lstlisting}

    Istotne jest to, że warunek sprawdzany jest \emph{przed} wykonaniem
    instrukcji.
\end{frame}

\begin{frame}[fragile]
    \frametitle{Inicjalizacja licznika}
    \framesubtitle{Pętla {\tt for}}

    Licznik jest inicjalizowany \emph{wewnątrz} pętli, wewnątrz nawiasów po
    słowie kluczowym {\tt for}:

    \begin{lstlisting}
    `std::cout << argv[0];`
    for (@auto i = 1@; i < argc; ++i) `{
        std::cout << " " << argv[i];
    }
    std::cout << "\n";`
    \end{lstlisting}
\end{frame}

\begin{frame}[fragile]
    \frametitle{Warunek}
    \framesubtitle{Pętla {\tt for}}

    Warunek zapisywany jest po średniku kończącym inicjalizację licznika:

    \begin{lstlisting}
    `std::cout << argv[0];`
    for (auto i = 1; @i < argc@; ++i) `{
        std::cout << " " << argv[i];
    }
    std::cout << "\n";`
    \end{lstlisting}

    Warunek, tak jak w pętli {\tt while}, jest sprawdzany przed wykonaniem
    instrukcji powtarzanej przez pętlę.
\end{frame}

\begin{frame}[fragile]
    \frametitle{Krok}
    \framesubtitle{Pętla {\tt for}}

    Krok jest wykonywany \emph{po} instrukcji powtarzanej w pętli, i służy do
    aktualizacji licznika:

    \begin{lstlisting}
    `std::cout << argv[0];`
    for (auto i = 1; i < argc; @++i@) `{
        std::cout << " " << argv[i];
    }
    std::cout << "\n";`
    \end{lstlisting}
\end{frame}

\begin{frame}[fragile]
    \frametitle{Instrukcja}
    \framesubtitle{Pętla {\tt for}}

    Instrukcja powtarzana przez pętlę jest zapisywana w nawiasach klamrowych:

    \begin{lstlisting}
    `std::cout << argv[0];
    for (auto i = 1; i < argc; ++i) `{
        @std::cout << " " << argv[i];@
    }
    `std::cout << "\n";`
    \end{lstlisting}
\end{frame}

\begin{frame}[fragile]
    \frametitle{Krok po kroku}
    \framesubtitle{Pętla {\tt for}}

    \begin{lstlisting}
    for
    \end{lstlisting}
    \vspace{2.4em}
\end{frame}

\begin{frame}[fragile]
    \frametitle{Krok po kroku}
    \framesubtitle{Pętla {\tt for}}

    \begin{lstlisting}
    `for` (auto i = 1;;)
    \end{lstlisting}
    \vspace{2.4em}
\end{frame}

\begin{frame}[fragile]
    \frametitle{Krok po kroku}
    \framesubtitle{Pętla {\tt for}}

    \begin{lstlisting}
    `for` (`auto i = 1`; i < argc;)
    \end{lstlisting}
    \vspace{2.4em}
\end{frame}

\begin{frame}[fragile]
    \frametitle{Krok po kroku}
    \framesubtitle{Pętla {\tt for}}

    \begin{lstlisting}
    `for` (`auto i = 1`; `i < argc`; ++i)
    \end{lstlisting}
    \vspace{2.4em}
\end{frame}

\begin{frame}[fragile]
    \frametitle{Krok po kroku}
    \framesubtitle{Pętla {\tt for}}

    \begin{lstlisting}
    `for (auto i = 1; i < argc; ++i)` {
        std::cout << " " << argv[i];
    }
    \end{lstlisting}
\end{frame}

\subsection{zadania}

\begin{frame}[fragile]
    \frametitle{Zadanie: hasło}
    \label{lecture_exercise_0}

    Program, który jako argument na wierszu poleceń pobierze napis (hasło), a
    potem będzie użytkownika prosił w pętli o ponowne podanie tego hasła dopóki
    nie zostanie ono wpisane poprawnie. Dla przykładu\footnote{{\tiny na zielono
    rzeczy wpisywane przez użytkownika}}:

    \begin{lstlisting}
    ./build/s03-password.bin student
    password: $profesor$
    password: $dziekan$
    password: $student$
    ok!
    \end{lstlisting}

    Kod źródłowy w pliku {\tt src/s03-password.cpp}
\end{frame}

\begin{frame}[fragile]
    \frametitle{Zadanie: odliczanie}
    \label{lecture_exercise_1}

    Program, który jako argument na wierszu poleceń pobierze liczbę i rozpocznie
    odliczanie od niej (włącznie) do zera (włącznie). Dla przykładu:

    \begin{lstlisting}
    ./build/s03-countdown.bin 3
    3...
    2...
    1...
    0...
    \end{lstlisting}

    Kod źródłowy w pliku {\tt src/s03-countdown.cpp}
\end{frame}

\begin{frame}[fragile]
    \frametitle{Zadanie: gra w zgadywanie}
    \label{lecture_exercise_2}

    Program, który wylosuje\footnote{patrz slajd 44. z pierwszego wykładu}
    liczbę całkowitą od 1 do 100 i będzie prosić użytkownika o zgadnięcie tej
    liczby. Po nieudanej próbie program powinien wyświetlić wskazówkę (np. ''za
    mała liczba'', ''za duża liczba'').

    \begin{lstlisting}
    ./build/s03-guessing-game.bin
    guess: $10$
    number too small!
    guess: $90$
    number too big!
    guess: $50$
    just right!
    \end{lstlisting}

    Kod źródłowy w pliku {\tt src/s03-guessing-game.cpp}
\end{frame}

\begin{frame}[fragile]
    \frametitle{Zadanie: FizzBuzz}
    \label{lecture_exercise_3}

    Program, który wczyta podaną jako argument na wierszu poleceń liczbę, a
    następnie dla każdego $n$ w zakresie od 1 (włącznie) do tej liczby
    (włącznie) wykona następujące rzeczy:

    \begin{enumerate}
        \item wypisze $n$ na ekran
        \item jeśli $n$ jest podzielne przez 3 wypisze ''Fizz'' (np. {\tt 3
            Fizz})
        \item jeśli $n$ jest podzielne przez 5 wypisze ''Buzz'' (np. {\tt 5
            Buzz})
        \item jeśli $n$ jest podzielne przez 3 i 5 wypisze ''FizzBuzz'' (np.
            {\tt 15 FizzBuzz})
    \end{enumerate}

    To czy liczba $a$ jest podzielna przez $n$ można sprawdzić operatorem {\tt
    \%} (\emph{modulo}) zwracającym resztę z dzielenia; `{\tt a \% n}' zwróci
    resztę z dzielenia $a$ przez $n$.

    Kod źródłowy w pliku {\tt src/s03-fizzbuzz.cpp}
\end{frame}

\begin{frame}[fragile]
    \frametitle{Zadanie: {\tt echo(1)}}
    \label{lecture_exercise_4}

    Program, który wypisze argumenty podane mu na wierszu poleceń. Wypisane
    argumenty muszą być odzielone znakiem spacji.

    Kod źródłowy w pliku {\tt src/s03-echo.cpp}

    \vspace{1em}

    {\small
    Ćwiczenie dodatkowe:
    \begin{enumerate}
        \item jeśli na początku pojawi się opcja {\tt -n} nie drukować znaku
            nowej linii na końcu programu
        \item jeśli na początku pojawi się opcja {\tt -r} wydrukować
            argumenty w odwrotnej kolejności
        \item jeśli na początku pojawi się opcja {\tt -l} wydrukować argumenty
            po jednym na linię
        \item obsłużyć sytuację, w której jednocześnie podane są opcje {\tt -r
            -l} albo {\tt -r -n}
    \end{enumerate}}
\end{frame}

\begin{frame}[fragile]
    \frametitle{Zadanie: 99 Bottles of beer}
    \label{lecture_exercise_5}

    Program powinien w pętli wyświetlić tekst
    piosenki\footnote{\url{https://en.wikipedia.org/wiki/99_Bottles_of_Beer}}.
    Rozpoczynając od 99 (lub liczby podanej jako argument na wierszu poleceń)
    program ma wypisać:

    \emph{99 bottles of beer on the wall, 99 bottles of beer.\newline
    Take one down, pass it around, 98 bottles of beer on the wall...}

    \vspace{1em}

    Po osiągnięciu 0 program ma wypisać:

    \emph{No more bottles of beer on the wall, no more bottles of beer.\newline
    Go to the store and buy some more, 99 bottles of beer on the wall...}

    i zakończyć pracę.

    \vspace{1em}

    Kod źródłowy w pliku {\tt src/s03-beer.cpp}
\end{frame}

\section{range-based {\tt for}}

\begin{frame}[fragile]
    \frametitle{Po co?}
    \framesubtitle{Pętla \emph{range-based} {\tt for}}

    Pętla \emph{range-based} {\tt for} sprawdza się kiedy instrukcja przez nią
    wykonywana powinna być powtórzona dla \emph{każdego elementu} pewnej
    wartości.

    \begin{lstlisting}
    for (auto const& each : employees) {
        pay_salary(each);
    }
    \end{lstlisting}

    Kompilator języka C++ automatycznie wygeneruje kod, który będzie
    odpowiedzialny za sprawdzenie warunku końca iteracji.
\end{frame}

\begin{frame}[fragile]
    \frametitle{Element}
    \framesubtitle{Pętla \emph{range-based} {\tt for}}

    Zmienna (lub stała), która reprezentuje aktualny element definiowana jest
    \emph{wewnątrz} pętli, wewnątrz nawiasów po słowie kluczowym {\tt for}:

    \begin{lstlisting}
    for (@auto const& each@ : employees) `{
        pay_salary(each);
    }`
    \end{lstlisting}

    {\footnotesize
    Jeśli pętla ma za zadanie zmodyfikować elementy trzeba użyć zapisu {\tt
    $T$\&}
    czyli \emph{referencja~do~$T$}\footnote{referencja to taki wskaźnik, który
    udaje, że nie jest wskaźnikiem i poprawia komfort życia programisty}.

    Jeśli modyfikowacja elementów jest niepożądana, warto użyć zapisu
    {\tt $T$ const}, czyli \emph{stała~typu~$T$}. Kompilator nie pozwoli na
    modyfikację takich wartości.

    Jeśli tworzenie kopii elementów jest kosztowne, a ich modyfikacje
    niepożądane można połączyć te dwa zapisy w {\tt$T$ const\&}, czyli
    \emph{referencja~do~stałej~typu~$T$}.
    }
\end{frame}

\begin{frame}[fragile]
    \frametitle{Zakres}
    \framesubtitle{Pętla \emph{range-based} {\tt for}}

    Zakres iteracji jest określony przez pewną wartość, podaną po dwukropku:

    \begin{lstlisting}
    for (auto const& each : @employees@) `{
        pay_salary(each);
    }`
    \end{lstlisting}

    Wartość ta może być zmienną, stałą, a nawet być wynikiem wywołania funkcji.
\end{frame}

\begin{frame}[fragile]
    \frametitle{Instrukcja}
    \framesubtitle{Pętla \emph{range-based} {\tt for}}

    Instrukcja podawana jest w nawiasach klamrowych i wykorzystuje element:

    \begin{lstlisting}
    `for (auto const& each : employees)` {
        pay_salary(each);
    }
    \end{lstlisting}
\end{frame}

\begin{frame}[fragile]
    \frametitle{Krok po kroku}
    \framesubtitle{Pętla \emph{range-based} {\tt for}}

    \begin{lstlisting}
    for
    \end{lstlisting}
    \vspace{2.4em}
\end{frame}

\begin{frame}[fragile]
    \frametitle{Krok po kroku}
    \framesubtitle{Pętla \emph{range-based} {\tt for}}

    \begin{lstlisting}
    `for` (auto const& each_element : )
    \end{lstlisting}
    \vspace{2.4em}
\end{frame}

\begin{frame}[fragile]
    \frametitle{Krok po kroku}
    \framesubtitle{Pętla \emph{range-based} {\tt for}}

    \begin{lstlisting}
    `for` (`auto const& each_element` : some_value)
    \end{lstlisting}
    \vspace{2.4em}
\end{frame}

\begin{frame}[fragile]
    \frametitle{Krok po kroku}
    \framesubtitle{Pętla \emph{range-based} {\tt for}}

    \begin{lstlisting}
    `for (auto const& each_element : some_value)` {
        use_element_to_do_stuff(each_element);
    }
    \end{lstlisting}
\end{frame}

\begin{frame}[fragile]
    \frametitle{Najprostszy przykład z możliwych}
    \framesubtitle{Pętla \emph{range-based} {\tt for}}

    {\footnotesize
    \begin{lstlisting}
    #include <algorithm>
    #include <iostream>
    #include <iterator>
    #include <string>
    #include <vector>

    auto main(int argc, char* argv[]) -> int
    {
        auto args = std::vector<std::string>{};
        std::copy_n(argv, argc, std::back_inserter(args));

        for (auto const& each : args) {
            std::cout << each << "\n";
        }

        return 0;
    }
    \end{lstlisting}}
\end{frame}

\begin{frame}[fragile]
    \frametitle{Najprostszy przykład z możliwych}
    \framesubtitle{Pętla \emph{range-based} {\tt for}}

    Ten fragment tworzy
    \emph{wektor}\footnote{\url{https://en.cppreference.com/w/cpp/container/vector}}
    z tablicy argumentów, przekazanych funkcji {\tt main()} na wierszu poleceń.

    {\footnotesize
    \begin{lstlisting}
    auto args = std::vector<std::string>{};
    std::copy_n(argv, argc, std::back_inserter(args));
    \end{lstlisting}}

    Funkcja {\tt std::copy\_n} (kopiująca {\tt argc} elementów z tablicy {\tt
    argv}) pochodzi z nagłówka {\tt algorithm}, a funkcja {\tt
    std::back\_inserter} (dodająca elementy do {\tt args}) z nagłówka
    {\tt iterator}.
\end{frame}

\begin{frame}[fragile]
    \frametitle{Najprostszy przykład z możliwych}
    \framesubtitle{Pętla \emph{range-based} {\tt for}}

    Następnie argumenty zebrane w zmiennej {\tt args} wypisywane są po kolei na
    standardowy strumień wyjścia.

    {\footnotesize
    \begin{lstlisting}
    for (auto const& each : args) {
        std::cout << each << "\n";
    }
    \end{lstlisting}}
\end{frame}

\subsection{zadania}

\begin{frame}[fragile]
    \frametitle{Kalkulator}
    \framesubtitle{Pętla \emph{range-based} {\tt for}}

    Używając pętli \emph{range-based} {\tt for} oraz biblioteki standardowej
    można szybko napisać własny kalkulator obliczający wyrażenia zapisane w
    \emph{odwrotnej notacji
    polskiej}\footnote{\url{https://en.wikipedia.org/wiki/Reverse_Polish_notation}}:

    \begin{lstlisting}
    make build/04-rpn-calculator.bin
    ./build/04-rpn-calculator.bin 2 2 + p
    4
    \end{lstlisting}

    {\tiny
    Kod źródłowy kalkulatora znajduje się w repozytorium z szablonem zajęć:
    \url{https://git.sr.ht/~maelkum/education-introduction-to-programming-cxx/tree/master/src/04-rpn-calculator.cpp}}
\end{frame}

\begin{frame}
    \frametitle{Kalkulator -- odwrotna notacja polska}
    \framesubtitle{Pętla \emph{range-based} {\tt for}}

    Odwrotna notacja polska jest notacją \emph{postfiksową} (ang.
    \emph{postfix}), czyli \emph{operator} występuje po \emph{operandach}:
    {\tt 2 2 +}\\
    Typowa, znana ze szkolnej matematyki, notacja z operatorem pomiędzy
    operandami to notacja infiksowa (ang. \emph{infix}): {\tt 2 + 2}\\
    Wywołania funkcji są zapisywane w notacji polskiej, prefiksowej (ang.
    \emph{prefix}) - czyli z operatorem zapisywanym przed operandami: {\tt + 2
    2}
\end{frame}

\begin{frame}
    \frametitle{Zadanie}
    \framesubtitle{Pętla \emph{range-based} {\tt for}}
    \label{lecture_exercise_6}

    Rozwinąć kalkulator z poprzednich slajdów o funkcje:

    {\footnotesize
    \begin{enumerate}
        \item mnożenia, operatorem {\tt *}
        \item dzielenia, operatorem {\tt /}
        \item dzielenia liczb całkowitych, operatorem {\tt //} (czyli `{\tt 5 2
            //}' da 2, a nie 2.5)
        \item reszty z dzielenia, operatorem {\tt \%}
        \item potęgowania, operatorem {\tt **}\footnote{operatory zawierające
            znak * trzeba na wierszu poleceń ''otoczyć'' znakiem apostrofu żeby
            powłoka nie potraktowała ich jako znaków specjalnych, np. {\tt 2 2
            '*'}}
        \item pierwiastka kwadratowego, operatorem {\tt sqrt}
        \item jednej operacji wymyślonej przez siebie
    \end{enumerate}}

    Kod źródłowy w pliku {\tt src/s04-rpn-calculator.cpp}
\end{frame}

\section{Podsumowanie}

\begin{frame}
    \frametitle{Co nowego?}
    \frametitle{Podsumowanie}

    Student powinien umieć:

    \begin{enumerate}
        \item wykorzystać pętle {\tt while}, {\tt do-while}, {\tt for}, oraz
            \emph{range-based {\tt for}}
        \item wykorzystać liczby losowe w programie
    \end{enumerate}
\end{frame}

\begin{frame}
    \frametitle{Zadania}
    \framesubtitle{Podsumowanie}

    Zadania znajdują się na slajdach
    \ref{lecture_exercise_0},
    \ref{lecture_exercise_1},
    \ref{lecture_exercise_2},
    \ref{lecture_exercise_3},
    \ref{lecture_exercise_4},
    \ref{lecture_exercise_5},
    \ref{lecture_exercise_6}.
\end{frame}

\end{document}
