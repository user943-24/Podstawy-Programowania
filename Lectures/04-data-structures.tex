\documentclass[aspectratio=169,10pt]{beamer}

\usepackage[utf8]{inputenc}
\usepackage{textcomp}
\usepackage[polish]{babel}
\usepackage{amsthm}
\usepackage{graphicx}
\usepackage[T1]{fontenc}
\usepackage{scrextend}
\usepackage{hyperref}
\usepackage{xcolor}
\usepackage{geometry}
\usepackage{listings}
\usepackage{epigraph}

\renewcommand{\epigraphsize}{\scriptsize}

\usetheme{-bjeldbak/beamerthemebjeldbak}

\definecolor{xbarcolor}{HTML}{000000}
\setbeamercolor{lower separation line head}{bg=xbarcolor}
\setbeamercolor{lower separation line foot}{bg=xbarcolor}

\title{Podstawy programownia (w języku C++)}
\subtitle{Struktury danych}
\author{Marek Marecki}
\institute{Polsko-Japońska Akademia Technik Komputerowych}

\lstset{basicstyle=\ttfamily\color{black},
columns=fixed,
escapeinside={[*}{*]},
inputencoding=utf8,
extendedchars=true,
moredelim=**[is][\color{red}]{@}{@},
moredelim=**[is][\color{gray}]{`}{`},
moredelim=**[is][\color{olive}]{$}{$}}

\begin{document}

{%
    \setbeamertemplate{headline}{}
    \frame{\titlepage}
}

\section{Struktury danych}

\begin{frame}
    \frametitle{Po co?}
    \framesubtitle{Struktury danych}

    Struktury (typy) danych wykorzystuje się jeśli zachodzi potrzeba zgrupowania
    pewnych danych mających \emph{wspólny raison
    d'être\footnote{\url{https://en.wiktionary.org/wiki/raison_d'être}}};
    reprezentujących zestaw \emph{danych} i \emph{operacji} na tych danych,
    których przechowywanie osobno nie miałoby większego sensu.

    % \vspace{1em}

    % Biblioteka standardowa zawiera strukturę danych {\tt std::vector}, która
    % reprezentuje tablicę o zmiennej długości. Grupuje ona \emph{dane} (zawartość
    % i rozmiar tablicy), oraz \emph{operacje} (np.~dodawanie, usuwanie i dostęp do
    % elementów).
\end{frame}

\begin{frame}[fragile]
    \frametitle{Wyliczenia}
    \framesubtitle{Struktury danych}

    Pierwszym rodzajem struktur danych w C++ są typy wliczeniowe.\\
    Są ich dwa rodzaje -- silne (\texttt{enum class}) i słabe (\texttt{enum}):

    {\scriptsize
    \begin{lstlisting}
    enum class Strong {
        CONAN_THE_BARBARIAN,
        JOHN_CARTER_OF_MARS,
    };

    enum Weak {
        KOBOLD,
        GNOLL,
    };
    \end{lstlisting}}

    Wyliczenia silne definiują nowy typ danych i pozwalają jedynie na
    porównywanie wartości.
    Słabe są \emph{de facto} wartościami typu \texttt{int} w przebraniu, i
    pozwalają na wszystko na co pozwala \texttt{int} (w tym automatyczne
    konwersje między wartościami typu \texttt{int} i słabymi wyliczeniami).
\end{frame}

\begin{frame}[fragile]
    \frametitle{Struktury}
    \framesubtitle{Struktury danych}

    Drugim rodzajem struktur danych w C++ są struktury.

    {\scriptsize
    \begin{lstlisting}
    struct Player_character {
        std::string name;
        short int   proficiency_bonus;
        int         hit_points;
        short int   hit_dice;
    };
    \end{lstlisting}}

    Definiują one nowy typ danych i pozwalają na definiowanie dla nich dowolnych
    operacji.
\end{frame}

\subsection{Wyliczenie}

\begin{frame}
    \frametitle{Składniki}
    \framesubtitle{Wyliczenia}

    Wyliczenia jawnie definiują wartości jakie mogą przybrać.

    Silny typ wyliczeniowy reprezentuje logiczną sumę (alternatywę, \emph{albo}) wszystkich
    swoich wartości: wartość A \emph{albo} wartość B, nigdy obie naraz.

    Słaby typ wyliczeniowy to nieco bardziej śliski temat, i może ``mieszać''
    deklarowane przez siebie wartości -- jest w stanie reprezentować wartość,
    która nie jest jawnie wyliczona.
\end{frame}

\begin{frame}[fragile]
    \frametitle{Silne wyliczenia}
    \framesubtitle{Wyliczenia}

    Przydatne do zdefiniowania dyskretnych, ``niemieszalnych'' wartości:

    {\scriptsize
    \begin{lstlisting}
    enum class Emotion {
        ANGER,
        FRUSTRATION,
        FEAR,
        JOY,
        HOPE,
    };

    auto es = std::set<Emotion>{ Emotion::FEAR, Emotion::HOPE };
    \end{lstlisting}}

    Jeśli zmienna powinna być w stanie reprezentować kilka takich wartości naraz
    można użyć \texttt{std::set} (z nagłówka \texttt{<set>}) do ich przechowania.
\end{frame}

\begin{frame}[fragile]
    \frametitle{Słabe wyliczenia}
    \framesubtitle{Wyliczenia}

    Przydatne do zdefiniowania ``mieszalnych'' wartości lub flag:

    {\scriptsize
    \begin{lstlisting}
    enum Emotion {
        ANGER       = (1 << 0),
        FRUSTRATION = (1 << 1),
        FEAR        = (1 << 2),
        JOY         = (1 << 3),
        HOPE        = (1 << 4),
    };

    auto ew = FEAR | HOPE;
    \end{lstlisting}}

    Mieszanie wartości można osiągnąć za pomocą operatora sumy
    bitowej\footnote{ale wtedy sztuczka z przesunięciami bitowymi (ang.
    \emph{bit shift}) jest konieczna, bo każde pole musi mieć zapalony inny bit}
    czyli \texttt{|} (tzw. \emph{pipe}).
\end{frame}

\begin{frame}[fragile]
    \frametitle{Wyliczenia silne \emph{vs} słabe}
    \framesubtitle{Wyliczenia}

    Reprezentacja kilku wartości naraz:

    {\scriptsize
    \begin{lstlisting}
    $// enumeration, strong$
    auto es = std::set<Emotion>{ Emotion::FEAR, Emotion::HOPE };

    $// add new value to the mix$
    es.insert(Emotion::ANGER);
    \end{lstlisting}}

    vs

    {\scriptsize
    \begin{lstlisting}
    $// enumeration, weak$
    auto ew = FEAR | HOPE;

    $// add new value to the mix$
    ew |= ANGER;
    \end{lstlisting}}
\end{frame}

\begin{frame}[fragile]
    \frametitle{Wyliczenia silne \emph{vs} słabe}
    \framesubtitle{Wyliczenia}

    Sprawdzenie czy zmienna zawiera daną wartość:

    {\scriptsize
    \begin{lstlisting}
    auto es = std::set<Emotion>{ Emotion::FEAR, Emotion::HOPE };

    if (es.count(Emotion::ANGER)) { $/* ... */$ }
    \end{lstlisting}}

    vs

    {\scriptsize
    \begin{lstlisting}
    auto ew = FEAR | HOPE;

    if (ew & ANGER) { $/* ... */$ }
    \end{lstlisting}}
\end{frame}

\begin{frame}[fragile]
    \frametitle{Wyliczenia silne \emph{vs} słabe}
    \framesubtitle{Wyliczenia}

    Usunięcie wartości ze zmiennej:

    {\scriptsize
    \begin{lstlisting}
    auto es = std::set<Emotion>{ Emotion::FEAR, Emotion::HOPE };

    es.erase(Emotion::FEAR);
    \end{lstlisting}}

    vs

    {\scriptsize
    \begin{lstlisting}
    auto ew = FEAR | HOPE;

    ew = (ew ^ FEAR);
    \end{lstlisting}}
\end{frame}

\begin{frame}[fragile]
    \frametitle{Wyliczenia silne \emph{vs} słabe}
    \framesubtitle{Wyliczenia}

    Wyliczenia silne i słabe pozwalają na osiągnięcie tego samego celu, ale na
    różne sposoby.

    \vspace{1em}

    Zaletą wyliczeń silnych jest to, że definiują nowy typ danych więc nie ma
    szans na przypadkową automatyczną konwersję, która sprawi, że w programie
    pojawi się ``niewyliczona'' wartość.
\end{frame}

\subsection{Struktura}

\begin{frame}[fragile]
    \frametitle{Składniki (1)}
    \framesubtitle{Struktury}

    Struktury (typy) danych składają się przede wszystkim z \textbf{pól} (ang.
    \emph{fields} lub \emph{member variables}), czasem zwanych zmiennymi lub
    stałymi składowymi.

    Typ strukturalny reprezentuje logiczny iloczyn (koniunkcję, \emph{i})
    wartości reprezentowalnych przez typy swoich pól: wartość A \emph{i} wartość
    B.

    \vspace{1em}

    Struktury mogą też zawierać \textbf{funkcje składowe} (ang. \emph{member
    functions} lub \emph{methods}) czasem zwanych metodami.
\end{frame}

\begin{frame}[fragile]
    \frametitle{Składniki (2)}
    \framesubtitle{Struktury}
    \label{struct_cat_example_full}

    {\scriptsize
    \begin{lstlisting}
    struct Cat {
        size_t number_of_lives { 9 };     $// member variable (field)$
        bool const brings_luck { false }; $// member constant (field)$

        auto make_sound(bool const) const -> void;  $// member function$
    };

    auto Cat::make_sound(bool const loud) const -> void
    {
        std::cout << (loud ? "MEOW!!!" : "Meow!")
            << " I have " << number_of_lives << " lives left, and I "
            << (brings_luck ? "do" : "don't") << " bring luck.\n";
    }
    \end{lstlisting}}
\end{frame}

\begin{frame}[fragile]
    \frametitle{Pola}
    \framesubtitle{Struktury}

    Pola służą do przechowywania \emph{danych}, które dany typ grupuje -- na
    przykład rozmiar i zawartość wektora w \texttt{std::vector}.

    \vspace{1em}

    Pola mogą być zmienne - jeśli ich wartości mogą podlegać modyfikacji w
    trakcie życia obiektu danego typu (np. rozmiar {\tt std::vector}), albo
    stałe - jeśli nie podlegają modyfikacji (np. rozmiar {\tt std::array}).

    \vspace{1em}

    Definiując pole trzeba użyć starego stylu deklaracji, czyli poprzedzić nazwę
    typem, a nie słowem kluczowym {\tt auto}:

    {\scriptsize
    \begin{lstlisting}
    size_t number_of_lives { 9 };
    \end{lstlisting}}
\end{frame}

\begin{frame}
    \frametitle{Funkcje składowe}
    \framesubtitle{Struktury}

    Funkcje składowe służą do wykonywania \emph{operacji} na danych zgrupowanych
    przez dany typ (np. {\tt std::vector::push\_back}), lub ogólnej
    interakcji z nim (np. {\tt Cat::make\_sound}).

    \vspace{1em}

    Funkcja składowa ma dostęp zarówno do wartości przekazanych jej w
    argumentach, jak i do wartości pól obiektu, na którym jest wywoływana (patrz
    przykład kodu na slajdzie \pageref{struct_cat_example_full}).

    \vspace{1em}

    Funkcje składowe mogą też pełnić specjalną rolę, np. konstruktor lub
    przeładowany operator (patrz slajdy \pageref{ctor_example} i
    \pageref{overloaded_operator_example}).
\end{frame}

\begin{frame}[fragile]
    \frametitle{{\tt class} vs {\tt struct}}
    \framesubtitle{Struktury}

    \emph{Jedyna} różnica między strukturami (\texttt{struct}), a klasami
    (\texttt{class}) to domyślna \emph{widoczność} pól i funkcji składowych.
    Pola klas są prywatne, a struktur publiczne.
    Widoczność można zmieniać słowami kluczowymi {\tt private} i {\tt public}:

    {\scriptsize
    \begin{lstlisting}
    struct S {
        bool now_you_see_me { true };

    @private@:
        bool now_you_dont { false };
    };

    class C {
        bool you_dont_see_me { false };

    @public@:
        bool now_you_do { true };
    };
    \end{lstlisting}}
\end{frame}

\begin{frame}[fragile]
    \frametitle{Widoczność pól}
    \framesubtitle{Struktury}

    Po co jest widoczność? Niektóre pola w strukturze mogą nie być przeznaczone
    dla ''użytkowników'' lub wymagać zachowania pewnych warunków. Jeśli jest
    taka potrzeba zawsze można udzielić dostępu do pola przez funkcje składowe.

    {\tiny
    \begin{lstlisting}
    class Hour {
        unsigned value { 0 };  $// private by default$
    public:
        auto what_time_is_it() const -> unsigned;
        auto increase_time() -> void;
    };
    auto Hour::what_time_is_it() const -> unsigned
    {
        reurn value;
    }
    auto Hour::increase_time() -> void
    {
        ++value;
        if (value > 23) {  $// make sure the hour wraps after 23$
            value = 0;
        }
    }
    \end{lstlisting}}
\end{frame}

\section{Pola}

\begin{frame}[fragile]
    \frametitle{Zmienne}
    \framesubtitle{Pola}

    Definicja zmiennego pola wygląda tak jak definicja każdej innej zmiennej, z
    tym wyjątkiem, że trzeba użyć starego stylu deklaracji (tj. bez
    \texttt{auto}).

    Załóżmy, że chcemy stworzyć strukturę opisującą kota. Jak wiadomo, koty mają
    9 żyć do wykorzystania:

    {\small
    \begin{lstlisting}
    struct Cat {
        static constexpr unsigned MAX_LIVES = 9;
        unsigned lives_left     { MAX_LIVES };
    };
    \end{lstlisting}}
\end{frame}

\begin{frame}[fragile]
    \frametitle{Stałe}
    \framesubtitle{Pola stałe (stałe składowe)}

    Definicja stałego pola wygląda podobnie do definicji zmiennego pola. Trzeba
    dodać jedynie słowo kluczowe \texttt{const}. Kontynuując przykład z kotem:

    {\scriptsize
    \begin{lstlisting}
    struct Cat {
        static constexpr unsigned MAX_LIVES = 9;
        unsigned lives_left { MAX_LIVES };

        std::string @const@ name;
    };
    \end{lstlisting}}

    Wartości pół stałych nie można\footnote{Oh, rly? Ciekawostka na slajdzie
    \pageref{hackerman_const_field_mutation}.} zmienić, nawet w ciele
    konstruktora. Do ich inicjalizacji służy specjalna notacja (patrz następny
    slajd).
\end{frame}

\begin{frame}
    \frametitle{Inicjalizacja pól}
    \framesubtitle{Pola}

    Pola można zainicjalizować na kilka sposobów:
    \begin{enumerate}
        \item przypisując im wartość w liście inicjalizującej składowe (ang.
            \emph{member initialiser list}) (patrz slajd
            \pageref{ctor_field_init_example})
        \item przypisując im wartość w ciele konstruktora
        \item pozostawiając ich wartości domyślne zdefiniowane w ciele struktury
    \end{enumerate}

    \vspace{1em}

    Zmienne składowe można zainicjalizować na każdy z powyższych sposobów.

    Stałe składowe można zainicjalizować jedynie w sposób 1. lub 3.
\end{frame}

\begin{frame}[fragile]
    \frametitle{Inicjalizacja pól (c.d.)}
    \framesubtitle{Pola}
    \label{ctor_field_init_example}

    {\scriptsize
    \begin{lstlisting}
    struct Cat {
        $/* fields of Cat omitted */$

        Cat() = default; $// default ctor$
        Cat(std::string, unsigned const);
    };

    Cat::Cat(std::string n, unsigned const ll)
        : name{n}        $// only member initialiser list for constants$
        , lives_left{ll} $// for variables you can use member initialiser list...$
    {
        lives_left = ll; $//...or assignment in constructor's body$
    }
    \end{lstlisting}}

    Jeśli definiujemy konstruktor to kompilator nie wygeneruje konstruktora
    domyślnego (z pustą listą parametrów).

    Domyślny konstrukor można w takim wypadku zdefiniować ręcznie, lub wymusić
    jego automatyczne utworzenie za pomocą konstrukcji ``\texttt{=~default}''.
\end{frame}

\begin{frame}[fragile]
    \frametitle{Dostęp}
    \framesubtitle{Pola}

    Aby dostać się do pola struktury należy użyć operatora dostępu czyli {\tt .}
    (kropka).

    Jeśli pole jest zmienne, można je zmodyfikować lub przypisać mu całkiem nową
    wartość:

    {\small
    \begin{lstlisting}
    auto a_cat = Cat{};      $// an ordinary cat$
    a_cat@.@lives_left @-=@ 1;   $// the cat died and lost a life$
    a_cat@.@lives_left  @=@ 666; $// now it's a cat from hell$
    \end{lstlisting}}

    Stałych pól nie można modyfikować, a kompilator w takim przypadku wygeneruje
    błąd:

    {\small
    \begin{lstlisting}
    auto mr_snuggles = Cat{ 9, "Mr. Snuggles" };
    std::cout << mr_snuggles@.@name << "\n";
    mr_snuggles@.@name @=@ "Evil Elvis"; @// error!@
    \end{lstlisting}}
\end{frame}

\begin{frame}
    \frametitle{Dostęp - modyfikacja stałych składowych?!}
    \framesubtitle{Pola stałe (stałe składowe)}

    \begin{figure}[!htp]
        \centering
        \includegraphics[width=9cm]{hackerman}
        \caption{Hackerman\footnote{Kung Fury (2015), \url{https://www.imdb.com/title/tt3472226/}}}
    \end{figure}
\end{frame}

\begin{frame}[fragile]
    \frametitle{Dostęp - modyfikacja stałych składowych?!}
    \framesubtitle{Pola stałe (stałe składowe)}
    \label{hackerman_const_field_mutation}

    Jeśli chce się zaimponować cioci w odwiedzinach, albo babci na święta to
    można pokazać im jak \textsc{oszukać kompilator} i zmodyfikować
    \textsc{stałe składowe}!

    {\scriptsize
    \begin{lstlisting}
    `std::cout << mr_snuggles.name << "\n";`

    auto wait_its_illegal = &mr_snuggles.name; $// pointer to member$
    @*const_cast<std::string*>(wait_its_illegal)@ = "Evil Elvis";

    `std::cout << mr_snuggles.name << "\n";`
    \end{lstlisting}}

    {\tiny Przy okazji widać też jak można pobrać \emph{wskaźnik do składowej}
    (ang. \emph{pointer to member}). Nie jest to coś co często się przydaje, ale
    na pewno częściej niż modyfikacja stałych składowych.}
\end{frame}

\section{Funkcje składowe}

\begin{frame}[fragile]
    \frametitle{Definiowanie funkcji składowych}
    \framesubtitle{Funkcje składowe}

    Funkcje składowe należy \emph{zadeklarować} wewnątrz struktury, do której mają
    należeć.

    {\small
    \begin{lstlisting}
    struct Cat {
        $// declaration, usually in header file: Cat.h$
        auto make_sound(bool const) const -> std::string;
    };

    $// definition, usually in implementation file: Cat.cpp$
    auto Cat::make_sound(bool const loud) const -> std::string
    {
        $/* ... */$
    }
    \end{lstlisting}}

    \emph{Definicje} funkcji składowych znajdują się (zazwyczaj) poza definicją
    struktury.
\end{frame}

\begin{frame}[fragile]
    \frametitle{Wywoływanie funkcji składowych}
    \framesubtitle{Funkcje składowe}

    Wywoływanie funkcji składowych odbywa się podobnie jak wywoływanie funkcji
    wolnych (ang. \emph{free function}), z tym wyjątkiem, że ich pierwszym
    parametrem jest obiekt, \emph{na którym} są wywoływane. Obiekt ten pojawia
    się przed operatorem dostępu (tak jak w dostępie do pól).

    {\small
    \begin{lstlisting}
    auto a_cat = Cat{};
    `std::cout <<` a_cat@.@make_sound@(@false@)@ `<< "\n";`
    \end{lstlisting}}
\end{frame}

\begin{frame}[fragile]
    \frametitle{``Ten'' obiekt}
    \framesubtitle{Funkcje składowe}

    Funkcja składowa jako ukryty argument otrzymuje informację o tym \emph{na
    jakim obiekcie} została wywołana.

    \vspace{1em}

    Używając słowa kluczowego \texttt{this} można dostać się do wskaźnika do
    ``tego'' obiektu i jawnie go używać:

    {\small
    \begin{lstlisting}
    auto Cat::make_sound(bool const) const -> std::string
    {
        return ("I have " + std::to_string(@this@->lives_left)
            + " lives left.");
    }
    \end{lstlisting}}
\end{frame}

\subsection{Przeładowywanie operatorów}

\begin{frame}[fragile]
    \frametitle{Przeładowanie operatorów}
    \framesubtitle{Struktury}
    \label{overloaded_operator_example}

    W C++ możliwe jest ``przeładowanie operatorów'' (ang. \emph{operator
    overloading}) czyli zdefiniowanie dla struktur danych funkcji składowych
    wywoływanych np. przez operatory arytmetyczne \texttt{+} lub \texttt{-}.

    {\scriptsize
    \begin{lstlisting}
    struct Modulo_arithmetic {
        int const mod { std::numerical_limits<int>::max() };
        int value { 0 };

        Modulo_arithmetic(int v, int m): mod{m}, value{v} {}

        auto @operator+@ (Modulo_arithmetic const) const -> Modulo_arithmetic;
        auto @operator<@ (Modulo_arithmetic const) const -> bool;
    };
    \end{lstlisting}}
\end{frame}

\begin{frame}[fragile]
    \frametitle{Definiowanie przeładowanych operatorów}
    \framesubtitle{Struktury}

    Funkcja składowa implementująca operator poza tym, że ma ``śmieszną'' nazwę
    niczym się nie różni od zwykłej funkcji składowej:

    {\scriptsize
    \begin{lstlisting}
    auto Modulo_arithmetic::@operator+@ (Modulo_arithmetic const o) const
        -> Modulo_arithmetic
    {
        $// FIXME what if the mods are different?$
        return Modulo_arithmetic{(value + o.value) % mod, mod};
    }

    auto Modulo_arithmetic::@operator<@ (Modulo_arithmetic const o) const
        -> bool
    {
        return (value < o.value);
    }
    \end{lstlisting}}
\end{frame}

\subsection{Specjalne funkcje składowe}

% FIXME use this when talking about resources, I/O, and threads
% \begin{frame}
%     \frametitle{Destruktor}
%     \framesubtitle{Struktury danych}

%     (ang. \textit{destructor})
% \end{frame}

% \begin{frame}
%     \frametitle{Kopiowanie}
%     \framesubtitle{Struktury danych}

%     (ang. \textit{destructor})
% \end{frame}

% \begin{frame}
%     \frametitle{Przenoszenie}
%     \framesubtitle{Struktury danych}

%     (ang. \textit{destructor})
% \end{frame}

\begin{frame}[fragile]
    \frametitle{Konstruktor}
    \framesubtitle{Struktury}
    \label{ctor_example}

    Konstruktor (ang. \textit{constructor}) jest specjalną funkcją składową
    odpowiedzialną za ``przygotowanie'' struktury danych do użycia.

    {\scriptsize
    \begin{lstlisting}
    struct Hour {
        int value { 0 };

        explicit Hour(unsigned): value{v}
        {
            @if@ (@v > 23@) { $// throw if they don't make sense$
                @throw@ std::out_of_range{"hour value cannot exceed 23"};
            }
        }
    };
    \end{lstlisting}}

    Konstruktor może zasygnalizować niemożność utworzenia instancji struktury
    (np. z powodu niewłaściwych wartości argumentów) używając
    wyjątków\footnote{ponieważ konstruktor jako takie nie zwraca wartości więc
    nie ma jak inaczej zasygnalizować błędu}.
\end{frame}

\begin{frame}[fragile]
    \frametitle{Destruktor}
    \framesubtitle{Struktury}

    Destruktor (ang. \textit{destructor}) jest specjalną funkcją składową
    odpowiedzialną za ``zniszczenie'' struktury danych. Jest uruchamiany w
    momencie tuż przed końcem czasu życia obiektu.

    {\scriptsize
    \begin{lstlisting}
    struct Network_connection {
        ~Network_connection()
        {
            shutdown(sock, SHUT_RDWR);
            close(sock);
        }
    };
    \end{lstlisting}}

    Destruktory definiuje się zazwyczaj dla typów mających ``opakować'' jakiś
    zasób, na przykład~pamięć, uchwyt do pliku, lub połączenie sieciowe.
\end{frame}

\begin{frame}
    \frametitle{RAII}
    \framesubtitle{Struktury}

    RAII (ang. \emph{Resource Acquisition Is Initialisation}) jest metodą
    zarządzania czasem życia obiektów\footnote{język Rust zawiera podobny
    mechanizm przez swoje jawne śledzenia czasów życia -- czyli lifetimes} w
    języku C++, definiującą kiedy obiekty są tworzone, kopiowane, przenoszone,
    i niszczone. Gwarantuje ona automatyczne wywołanie konstruktora,
    destruktora, oraz innych funkcji zarządzających obiektami w odpowiednich
    momentach.

    \vspace{1em}

    Jest to kluczowa sprawa dla zapewnienia poprawnego zarządzania zasobami oraz
    poprawności programu.
\end{frame}

\begin{frame}
    \frametitle{``Rule of zero''}
    \framesubtitle{RAII}

    Zazwyczaj dla klasy definiuje się konstruktor, aby sterować tworzeniem jej
    wartości. Dla klas, które nie zarządzają zasobami zdefiniowanie konstruktora
    jest wystarczające i stosuje się ``Rule of zero'' czyli nie podawanie
    definicji żadnej z funkcji biorących udział w zarządzaniu czasem życia
    obiektów.
\end{frame}

\begin{frame}
    \frametitle{``Rule of five''}
    \framesubtitle{RAII}

    ``Rule of
    five''\footnote{\url{https://en.cppreference.com/w/cpp/language/rule_of_three}}
    mówi natomiast, że jeśli zdefiniowana jest jedna z tych funkcji, to należy
    najprawdopodobniej zdefiniować wszystkie pięć:

    \vspace{1em}

    \begin{labeling}{1. konstruktor przenoszący}
        \item[1. konstruktor kopiujący] \texttt{T(T const\&)}
        \item[2. konstruktor przenoszący] \texttt{T(T\&\&)}
        \item[3. kopiujący \texttt{operator=}] \texttt{operator=(T const\&) -> T\&}
        \item[4. przenoszący \texttt{operator=}] \texttt{operator=(T\&\&) -> T\&}
        \item[5. destruktor] \texttt{~T()}
    \end{labeling}
\end{frame}

\begin{frame}[fragile]
    \frametitle{``Rule of five'' (c.d.)}
    \framesubtitle{RAII}

    {\tiny
    \begin{lstlisting}
    class Network_connection {
        int sock { -1 };

        Network_connection(Network_connection const&) @= delete@;
        Network_connection(Network_connection&& o)
            : sock{std::move(o.sock)}
        {
            o.sock = -1;
        }

        auto operator=(Network_connection const&) @= delete@;
        auto operator=(Network_connection&& o)
        {
            sock = std::move(o.sock);
            o.sock = -1;
        }

        ~Network_connection()
        {
            shutdown(sock, SHUT_RDWR);
            close(sock);
        }
    };
    \end{lstlisting}}
\end{frame}

\section{Zadania}

\begin{frame}
    \frametitle{Zadanie: student}
    \label{lecture_exercise_0}

    Zaimplementować strukturę danych opisującą studenta. Struktura powinna
    składać się z:
    \begin{enumerate}
        \item pól (imię, nazwisko, numer indeksu, aktualny semest,
            średnia ocen)
        \item funkcji składowej `{\tt to\_string() const}' zwracającej {\tt
            std::string}, którym opisuje studenta
        \item konstruktora
    \end{enumerate}

    Niech w funkcji main będzie utworzony obiekt reprezentujący was, a na
    \texttt{std::cout} wydrukowany będzie wynik działania funkcji
    \texttt{Student::to\_string} na tym obiekcie.

    \vspace{1em}

    {\footnotesize
    Kod źródłowy w plikach {\tt include/s1234/Student.h} (nagłówek) i
    {\tt src/s03-Student.cpp} (implementacja i funkcja {\tt main}).}
\end{frame}

\begin{frame}
    \frametitle{Zadanie: czas}
    \label{lecture_exercise_1}

    Zaimplementować strukturę danych opisującą czas. Struktura powinna składać
    się z:
    \begin{enumerate}
        \item pól (godzina, minuta, sekunda)
        \item funkcji składowych:
            \begin{enumerate}
                \item `{\tt to\_string() const}' zwracającej {\tt
                    std::string} pokazującej czas w formacie {\tt HH:MM:SS}
                \item {\tt next\_hour()}, {\tt next\_minute()}, i {\tt
                    next\_second()} (wszystkie zwracające {\tt void})
                    zwiększających czas
            \end{enumerate}
        \item konstruktora
    \end{enumerate}

    Niech w funkcji \texttt{main} pojawi się kod pozwalający na zweryfikowanie
    działania waszej struktury danych dla godziny 23:59:59 (np. niech drukuje
    godzinę, zwiększy mintę, wydrukuje znowu, itd.).

    \vspace{1em}

    {\footnotesize
    Kod źródłowy w plikach {\tt include/s1234/Time.h} (nagłówek) i
    {\tt src/s03-Time.cpp} (implementacja i funkcja {\tt main}).}
\end{frame}

\begin{frame}
    \frametitle{Zadanie: pora dnia}
    \label{lecture_exercise_2}

    Do struktury opisującej czas dodać funkcję składową
    `{\tt time\_of\_day() const}' zwracającą porę dnia (rano, dzień, wieczór,
    noc). Pora dnia powinna być opisana typem wyliczeniowym {\tt enum class
    Time\_of\_day}.

    Napisać funkcję \texttt{to\_string(Time\_of\_day)} zwracającą
    \texttt{std::string}, która zamieni wartość wyliczeniową na napis.

    W funkcji main dodać kod pozwalający na zweryfikowanie działania dodanych
    funkcji.

    \vspace{1em}

    {\footnotesize
    Kod źródłowy w plikach {\tt include/s1234/Time.h} (nagłówek) i
    {\tt src/s03-Time.cpp} (implementacja i funkcja {\tt main}).}
\end{frame}

\begin{frame}[fragile]
    \frametitle{Zadanie: arytmetyka}
    \label{lecture_exercise_3}

    Do struktury opisującej czas dodać funkcje składowe:

    {\scriptsize
    \begin{lstlisting}
    auto operator+ (Time const&) const -> Time;
    auto operator- (Time const&) const -> Time;
    auto operator< (Time const&) const -> bool;
    auto operator> (Time const&) const -> bool;
    auto operator== (Time const&) const -> bool;
    auto operator!= (Time const&) const -> bool;
    \end{lstlisting}}

    Umożliwią one arytmetykę (dodawanie, odejmowanie, porównywanie, itd.) na czasie.

    Do funkcji \texttt{main} dodać kod, który pozwoli na zweryfikowanie
    działania.

    \vspace{1em}

    {\footnotesize
    Kod źródłowy w plikach {\tt include/s1234/Time.h} (nagłówek) i
    {\tt src/s03-Time.cpp} (implementacja i funkcja {\tt main}).}
\end{frame}

\begin{frame}[fragile]
    \frametitle{Zadanie: sekundy do północy}
    \label{lecture_exercise_4}

    Do struktury opisującej czas dodać funkcje składowe:

    {\scriptsize
    \begin{lstlisting}
    auto count_seconds() const -> uint64_t;
    auto count_minutes() const -> uint64_t;
    auto time_to_midnight() const -> Time;
    \end{lstlisting}}

    Funkcje \texttt{count\_seconds()} i \texttt{count\_minutes()} liczą sekundy
    \emph{od} godziny 00:00:00.\\
    Funkcja \texttt{time\_to\_midnight()} zwraca czas pozostały \emph{do} północy.

    Do funkcji \texttt{main} dodać kod, który pozwoli na zweryfikowanie
    działania.

    \vspace{1em}

    {\footnotesize
    Kod źródłowy w plikach {\tt include/s1234/Time.h} (nagłówek) i
    {\tt src/s03-Time.cpp} (implementacja i funkcja {\tt main}).}
\end{frame}

\begin{frame}
    \frametitle{Zadanie: ctor i dtor}
    \label{lecture_exercise_5}

    Zaimplementować strukturę, która w konstruktorze przyjmie wartość typu
    \texttt{std::string}, a w destruktorze wydrukuje ją na ekran poprzedzoną
    napisem \texttt{"DESTRUCTION!"}.

    \vspace{1em}

    {\footnotesize
    Kod źródłowy w pliku{\tt src/s04-ctor-dtor.cpp}}
\end{frame}

\begin{frame}
    \frametitle{Zadanie: this}
    \label{lecture_exercise_6}

    Zaimplementować strukturę, która będzie miała funkcję składową drukującą na
    ekran wartość wskaźnika \texttt{this}. W funkcji \texttt{main()} wywołać tą
    funkcję na stworzonym obiekcie, oraz pobrać jego adres ``ręcznie''
    operatorem \texttt{\&} i porównać wyniki.

    \vspace{1em}

    {\footnotesize
    Kod źródłowy w pliku{\tt src/s04-this.cpp}}
\end{frame}

\section{Podsumowanie}

\begin{frame}
    \frametitle{Co nowego?}
    \frametitle{Podsumowanie}

    Student powinien umieć:

    \begin{enumerate}
        \item samodzielnie zaprojektować własny typ danych, jego pola i funkcje
            składowe
        \item wytłumaczyć czym jest i jak działa funkcja składowa, oraz czym
            jest {\tt this}
        \item powiedzieć jaka jest rola konstruktora, destruktora
        \item znać pojęcia takie jak RAII, Rule of zero, Rule of five
    \end{enumerate}
\end{frame}

\begin{frame}
    \frametitle{Zadania}
    \framesubtitle{Podsumowanie}

    Zadania znajdują się na slajdach
    \ref{lecture_exercise_0},
    \ref{lecture_exercise_1},
    \ref{lecture_exercise_2},
    \ref{lecture_exercise_3},
    \ref{lecture_exercise_4},
    \ref{lecture_exercise_5}, i
    \ref{lecture_exercise_6}.
\end{frame}

\end{document}
